\section{Mora and Robianno Algorithm:
  Discussion and Implementation in Sage}

Here is an implementation \footnote{Comments in Python begin with \#} of algorithm
by Mora and Robbiano for computing \grob fan of an ideal:

\begin{lstlisting}[language=Python]
# 'inputBasis' is an array of polynomials
def groebnerFan(inputBasis):

    # Initialization
    L = ([], [], [], {}, [])
    Lnew = [L]
    for polynomial in inputBasis:
        Lold = Lnew
        Lnew = []
        for (G, M, E, Psi, B) in Lold:
            for leadingMonomial in polynomial.monomials():
                Gnew = G[:]
                Gnew.append(polynomial)
                Mnew = M[:]
                Mnew.append(leadingMonomial)
                Enew = E[:]
                for nonLeadingMonomial in polynomial.monomials():
                    if(nonLeadingMonomial != leadingMonomial):
                        # We substract the Leading Monomial to the
                        # Non Leading Monomials because the LP solver
                        # has <= as default inequalities
                        Enew.append(subtractExponents(nonLeadingMonomial,
                                                      leadingMonomial))
                Psinew = copy.deepcopy(Psi)
                Psinew[polynomial] = leadingMonomial
                Bnew = B[:]
                for g in G:
                    Bnew.append((g, polynomial))
                if isNotEmptyTO(Enew):
                    L = (Gnew, Mnew, Enew, Psinew, Bnew)
                    Lnew.append(L)
                    
    # Computation of the Groebner Bases
    Lwork = Lnew
    Lpartial = []
    while (Lwork != []):
        G, M, E, Psi, B = Lwork.pop()
        f, g = B.pop()
        T = lcm(Psi[f], Psi[g])
        gCoeffPsiG = g.monomial_coefficient(Psi[g])
        fCoeffPsiF = f.monomial_coefficient(Psi[f])
        h = gCoeffPsiG*T*f//Psi[f] - fCoeffPsiF*T*g//Psi[g]
        check, subtract = minimalPolynomialCheck(G, Psi, h)
        while check:
            h = h - subtract
            check, subtract = minimalPolynomialCheck(G, Psi, h)
        if h == 0 :
            if (B == []):
                Lpartial.append((G, M, E, Psi))
            else:
                Lwork.append((G, M , E, Psi, B))
        else:
            for leadingMonomial in h.monomials():
                Gnew = G[:]
                Gnew.append(h)
                Mnew = M[:]
                Mnew.append(leadingMonomial)
                Enew = E[:]
                for nonLeadingMonomial in h.monomials():
                    if(nonLeadingMonomial != leadingMonomial):
                        # We substract the Leading Monomial to the
                        # Non Leading Monomials because the LP solver
                        # has <= as default inequalities
                        Enew.append(subtractExponents(nonLeadingMonomial,
                                                      leadingMonomial))
                Psinew = copy.deepcopy(Psi)
                Psinew[h] = leadingMonomial
                Bnew = B[:]
                for g in G:
                    Bnew.append((g, h))
                if isNotEmptyTO(Enew):
                    Lwork.append((Gnew, Mnew, Enew, Psinew, Bnew))

    # Computation of the Reduced Groebner Bases
    # and of the Groebner Region
    Loutput = []
    Mon = []
    while (Lpartial != []):
        G, M, E, Psi = Lpartial.pop()
        if (not membershipIdealArrayTest(M, Mon)):
            polynomial, check = reducibilityCheck(G, M, Psi)
            while check:
                G.remove(polynomial)
                M.remove(Psi[polynomial])
                del Psi[polynomial]
                polynomial, check = reducibilityCheck(G, M, Psi)
            for g in G:
                G.remove(g)
                M.remove(Psi[g])
                gnew = red(G, M, g)
                coeffGNew = gnew.monomial_coefficient(Psi[g])
                gnew = 1/coeffGNew*gnew
                G.append(gnew)
                M.append(Psi[g])
                tempPsiG = Psi[g]
                del Psi[g]
                Psi[gnew] = tempPsiG
            E = []
            for g in G:
                for monomial in g.monomials():
                    if(monomial != Psi[g]):
                        E.append(subtractExponents(Psi[g],
                                                   monomial))
            Loutput.append((G, M, E, Psi))
            Mon.append(M)
    return Loutput
\end{lstlisting}

As mentioned before, the algorithm has three main components. The implementation
is straightforward once is understood the high level idea. Many additional methods
where needed to implement separately in order to provide a clean design.

It might be worth mentioning the main data structure used in the algorithm. In
many parts of the algorithm we will see the common decomposition of an array of
elements into the tuple $(G, M, E, Psi, B)$. These components stand for the following:

\begin{itemize}
\item $G$ is the set of polynomials for a \grob basis.
\item $M$ is the set of monomials keeping track of the leading monomials
  for the respective $G$.
\item $E$ is the set of inequalities produced by enumerating the constraints
  by $G$ and $M$.
\item $Psi$ is a map (dictionary structure in Python) that associates an element
  in $G$ with an element in $M$.
\item $B$ keeps track of the pair of elements in $G$ that have not reduced
  using Buchberger's criterion in order to avoid unnecessary computations.
\end{itemize}

In order to test if a set of inequalities define a monomial order (lines 29, 71)
we used the linear programming library `scipy.optimize.linprog' to find solution
to the set of inequalities. We realize that the library uses non-strict inequalities
and the lack of attention of the latter produced bugs in the program since the
linear solver was accepting set of inequalities that do not define monomial orders.
The latter was fixed by including a small epsilon value to the non-strict inequalities
to convert them into strict inequalities. 

Additionally, we used a theorem in \cite{Cox:2015:IVA:2821082}, to be more precise a
corollary about two monomial ideals being equivalent, to code line 80 in the previous
implementation.

%%% Local Variables:
%%% mode: latex
%%% TeX-master: "main"
%%% End:
