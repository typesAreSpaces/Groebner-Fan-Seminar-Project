\section{Introduction}

The Gr\"obner fan of an ideal was introduced in \cite{MORA1988183}.
This idea was motivated to study all possible reduced Gr\"obner
basis for all monomial orders since the latter heavily depends
on a particular monomial order \cite{Cox:2015:IVA:2821082}
since the main component in many Gr\"obner basis algorithms
relies on a division algorithm. A simple (and quite expected)
observation notices that different monomial orders lead
different Gr\"obner basis. Moreover, the performance of certain
algorithms is faster using certain monomial orders. In addition,
current complexity results \cite{MAYR1982305} indicate instances
where th runtime for ideals with polynomials of
degree at most $d$ is $\mathcal{O}(2^{2^d})$. Hence, it is interesting to
obtain a characterization of how to transform one Gr\"obner
basis to another with different monomial order due to the expensive
nature of the algorithm. For the latter,
some techniques which are based on the Gr\"obner fan, like Gr\"obner walk,
provide a solution. More efficient methods might rely on the use of
Universal Gr\"obner basis. However, the Gr\"obner fan has proven to
be useful in other areas of mathematics like tropical algebra, auction
design and optimization problems \cite{2014arXiv1408.0313K}.

Among of the main outcomes of \cite{MORA1988183} are:

\begin{itemize}
\item[1.] There is a one-to-one correspondence between reduced
  marked Gr\"obner bases with initial ideals.
\item[2.] The set of initial ideals is finite (hence the set of all
  reduced marked Gr\"obner bases is finite too).
\item[3.] For every initial ideals in $k[x_1, \dots, x_n]$
  there is a corresponding positive vector in $\real^n$.  
\end{itemize}

Similarly, in \cite{Cox:2014}, the authors motivated the above points
through three questions. An important fact mentioned
in both \cite{Cox:2014, SturmfelsGrobConv}, which received little attention,
was about the encoding of monomials orders. I consider this crucial since the
main focus of our study is to characterize Gr\"obner bases independently
of the monomial order, hence it is necessary to give a formal representation
of the latter. For instance in \cite{Cox:2014} this was achieved by
representing monomial orders using matrices, whereas \cite{SturmfelsGrobConv}
considered a recursive definition using \emph{vectors} and the standard \emph{dot product}
in $\integer^n$ assuming the existence of an arbitrary term order, which typically
is the lexicographical order. In the end, both approaches are useful
since the main property of these encodings is the relevance of the first
row of the matrix (initial vector respectively) to define an implicit
Gr\"obner basis, which is the so called \emph{marked \grob basis}.

Given all these properties, the algorithm by Mora and Robianno \cite{MORA1988183}
for computing \grob fans exploits the fact that the \grob basis are finite and so
a naive algorithm of three steps will always terminate. Their approach can be
studied as follows:

\begin{itemize}
\item[1.] Given a set of polynomials $S$, enumerating all possible initial monomials
  and filter the ones that are feasible using linear programming techniques.
  The enumeration is a combinatorial
  process of choosing a initial monomial $m$ for each polynomial $f$ in $S$ and making the
  correspondent inequalities with each non-initial monomial in $f$. 
\item[2.] Extend the initial set of polynomials to several marked \grob bases
  (each for every possible monomial order previously computed)
  using a \grob basis algorithm. This step is motivated by a result in
  \cite{Cox:2014} which states that the \grob fan covers the positive orthant
  of $\real^n$. Hence, if this step cannot find more extensions it means
  the set of marked \grob basis has already covered this section of $\real^n$.
\item[3.] Filter the previous marked \grob bases by computing reduced \grob
  bases. This step give us additionally a complete geometrically characterization
  of the reduced marked \grob bases and it is important since many of the marked
  \grob bases computed in the previous step might be the equivalent with different
  monomial orders. 
\end{itemize}

We will discuss several examples to illustrate the above algorithm in Section 4. In Section
2 we will discuss the relevance of the monomial orders, initial monomials, and its relevance
with the geometric structure that entails the finiteness of the construction \footnote{It is worth
  mentioning the set of monomial ideals is finite for the commutative case. However, for the non
  commutative case this construction is not finite as noted by Weispfenning in
  \cite{10.1007/3-540-51082-6_96}}. In Section 3 we will discuss an implementation of the
Mora and Robbiano algorithm using Python will additional support of the computer algebra system \emph{Sage}
\cite{SageMultivariatePolynomials} and the linear programming library scipy \cite{linprog}.


%%% Local Variables:
%%% mode: latex
%%% TeX-master: "main"
%%% End:
