\section{Introduction}

The Gr\"obner fan of an ideal was introduced in \cite{MORA1988183}.
This idea was motivated to study all possible reduced Gr\"obner
basis for all monomial orderings since the latter heavily depends
on a particular monomial ordering \cite{Cox:2015:IVA:2821082}
since the main component in many Gr\"obner basis algorithms
relies on a division algorithm. A simple (and quite expected)
observation notices that different monomial orderings lead
different Gr\"obner basis. Moreover, the performance of certain
algorithms is faster using certain monomial orderings. In addition,
current complexity results \cite{MAYR1982305} indicate instances
where th runtime for ideals with polynomials of
degree at most $d$ is $\mathcal{O}(2^{2^d})$. Hence, it is interesting to
obtain a characterization of how to transform one Gr\"obner
basis to another with different monomial ordering due to the expensive
nature of the algorithm. For ther latter,
some techniques which are based on the Gr\"obner fan, like Gr\"obner walk,
provide a solution. More efficient methods might rely on the use of
Universal Gr\"obner basis. However, the Gr\"obner fan has proven to
be useful in other areas of mathematics like tropical algebra, auction
design and optimization problems \cite{2014arXiv1408.0313K}.

Among of the main outcomes of \cite{MORA1988183} are:

\begin{itemize}
\item There is a one-to-one correspondance between reduced
  marked Gr\"obner bases with initial ideals.
\item The set of initial ideals is finite (hence the set of all
  reduced marked Gr\"obner bases is finite too).
\item For every initial ideals in $k[x_1, \dots, x_n]$
  there is a corresponding positive vector in $\real^n$.  
\end{itemize}

Similarly, in \cite{Cox:2014}, the authours motivated the above points
throught three questions. However, 

%%% Local Variables:
%%% mode: latex
%%% TeX-master: "main"
%%% End:
